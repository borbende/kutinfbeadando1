\documentclass[11pt]{beamer}
\usetheme{Warsaw}
\usepackage[utf8]{inputenc}
\usepackage[magyar]{babel}
\usepackage[T1]{fontenc}
\usepackage{amsmath}
\usepackage{amsfonts}
\usepackage{amssymb}
\usepackage{graphicx}
\usepackage{hyperref}
\usepackage{multimedia}
\author{Borkovits Bendegúz}
\title{Beamer}
%\setbeamercovered{transparent} 
%\setbeamertemplate{navigation symbols}{} 
%\logo{} 
\institute{ELTE} 
\date{2020} 
%\subject{} 
\begin{document}

\AtBeginSection[]
{
\begin{frame}
\frametitle{Tartalomjegyzék}
\tableofcontents[currentsection]
\end{frame}
}

\begin{frame}
\titlepage
\end{frame}

\section{Alapvető tudnivalók}

\subsection{Mi az a beamer?}

\begin{frame}
\frametitle{Mi ez?}
Lényegében egy \alert{LaTex} prezentációs eszköz.

Manuális beállítások

Customization over 9000!
\end{frame}

\begin{frame}
\frametitle{A frame}
\begin{block}{Mi az a slide?}
Egy frame sok  \pause
slide-ból is felépülhet. \pause
\end{block}
\begin{alertblock}{Tananyag mennyisége...}
Rendes nézetben a \alert{slide}-ok számát látjuk.
\end{alertblock}

\end{frame}

\subsection{Kinézet}

\begin{frame}
\frametitle{Kinézet}
\begin{block}{Stílus}
Ez a dokumentum Varsó stílusú.
\end{block}

\begin{block}{Beállítható:}
Stílus, kódolás, nyelv, usepackage...
\end{block}
\end{frame}

\section{Tutorial}

\subsection{Kezelőfelület és usepackage}

\begin{frame}
\frametitle{Környezet}
Alapverzió: Texworks

\begin{block}{Texmaker}
A LaTex szerkesztője, a dokumentumokat ezen a felületen hozom létre. Rengeteg funkció és felhasználóbarát. \alert{Editor} \pause
\end{block}

\begin{alertblock}{MikTex}
A Texmakerben létrehozott fájlok fordításáért felel. \alert{Compiler}
\end{alertblock}
\end{frame}

\begin{frame}{Első indítás}
\begin{block}{Usepackage letöltése}
Kell a nyelvhez, kódoláshoz, grafikához,stílushoz...
\end{block}

A legtöbb ember csak néhány megszokott usepackage-et használ.

Néhány speciális funkciónak megvan a maga usepackage-e.
\end{frame}

\subsection{Frame felépítés}

\begin{frame}[fragile]{Így írhatsz frame-et}
\begin{semiverbatim}
\\begin\{frame\}\{Környezet\}
Alapverzió: Texworks

\\begin\{block\}\{Texmaker\}
A LaTex szerkesztője, a dokumentumokat ezen a felületen 
hozom létre. Rengeteg funkció és felhasználóbarát.
 \\alert\{Editor\} \\pause
\\end\{block\}

\\begin\{alertblock\}\{MikTex\}
A Texmakerben létrehozott fájlok fordításáért felel.
 \\alert\{Compiler\}
\\end\{alertblock\}
\\end\{frame\}
\end{semiverbatim}
\end{frame}

\begin{frame}{Egyéb hasonló felépítésű parancsok}
\label{parancsok}
\begin{columns}

	\begin{column}{0.5\textwidth}
		\begin{itemize}
			\item Képbetöltés
			\begin{itemize}
				\item includegraphics
				\item figure
			\end{itemize}
			\item Listák
			\begin{enumerate}
				\item itemize
				\item enumerate
				\item description
			\end{enumerate}
			\item Táblázat
			\begin{enumerate}[I]
				\item table
				\item tabular
			\end{enumerate}
		\end{itemize}
	\end{column}

	\begin{column}{0.5\textwidth}
		\begin{itemize}
			\item Block
			\begin{itemize}
				\item block
				\item definition
				\item alertblock
				\item example
				\item theorem, corollary, proof
			\end{itemize}					
			\item Egyebek
			\begin{enumerate}[i]
				\item centering
				\item caption
				\item column
				\item semiverbatim
			\end{enumerate}
		\end{itemize}
	\end{column}

\end{columns}
\end{frame}

\begin{frame}{Táblázat}
	\begin{table}
		\begin{tabular}{l | c | c | c}
		Oktató & Kurzus & Nap & Idő \\
		\hline \hline
		Dávid Gyula & Még relatív & Szerda & 14:00 \\
		\hline
		Cserti József & Magic show & Csütörtök & 11:00 \\
		\hline
		Széchenyi Gábor & Műkorcsolya & Hétfő & 15:00 \\
		\end{tabular}
	\end{table}
\end{frame}

\begin{frame}{Tagolás}
\begin{columns}

	\begin{column}{0.5\textwidth}
		\begin{itemize}
			\item Tartalomjegyzék
			\begin{itemize}
				\item section
				\item subsection
			\end{itemize}
		\end{itemize}
	\end{column}

	\begin{column}{0.5\textwidth}
		\centering
		\includegraphics[scale=0.75]{section.png}
	\end{column}

\end{columns}
\end{frame}

\begin{frame}{Navigálás}
\begin{block}{Hiperhivatkozás}
Egyes framekre shortcutot készít.
\end{block}
\alert{Mi kell hozzá?}
\begin{semiverbatim}
\\label\{labelnév\}

\\hyperlink\{labelnév\}\{linkcím\}

\alert{vagy:}

\\hyperlink\{labelnév\}\{\\beamerbutton\{linkcím\}\}

\end{semiverbatim}
\begin{example}
\hyperlink{parancsok}{\beamerbutton{Milyen parancsokat használtam?}}
\end{example}
\end{frame}

\begin{frame}{Videó}
Ez egy picit komplikáltabb.

Legalább egy tucat különböző módszer.

\alert{Usepackage-ek}: media9, multimedia, movie15,...

\alert{Parancsok}: flashmovie, includemedia, includemovie, movie,...

Függhet a lejátszótól, környezettől, formátumtól,...
\end{frame}

\begin{frame}{Egyenletek}
Például a gravitációs erő nagysága: $F=\gamma*\frac{m_1*m_2}{r^2}$
\begin{block}{Usepackage:\alert{tikz}}
Képletekhez jelmagyarázatot készít, más szoftverekből berakhatunk ábrákat.
\end{block}
\end{frame}

\begin{frame}{Források (avagy URL tutorial)}
Alapvető ismeretek:\href{https://www.overleaf.com/learn/latex/Beamer}{\beamergotobutton{https://www.overleaf.com/learn/latex/Beamer}}

Tutorialok kezdőknek:\href{https://www.youtube.com/user/ShareLaTeX}{\beamergotobutton{https://www.youtube.com/user/ShareLaTeX}}

Texmaker letöltése:
\href{https://www.xm1math.net/texmaker/download.html}{\beamergotobutton{https://www.xm1math.net/texmaker/download.html}}

MikTex letöltése:
\href{https://miktex.org/download}{\beamergotobutton{https://miktex.org/download}}
\end{frame}

\section{Vélemény}

\begin{frame}{Ajánlás}
\begin{block}{Előny}
Kezdeti nehézségek után könnyebb kezelni, mint a ppt-t.
\end{block}

\begin{alertblock}{Hátrány}
De néhány része homályos és nem egyértelmű.
\end{alertblock}

\begin{block}{Vélemény}
Ideális a prezentációk elkészítéséhez.
\end{block}
\end{frame}
\end{document}